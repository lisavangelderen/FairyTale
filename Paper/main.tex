%
% File acl2014.tex
%
% Contact: g.colavizza@uva.nl
%%
%% Based on the style files for ACL-2013, which were, in turn,
%% Based on the style files for ACL-2012, which were, in turn,
%% based on the style files for ACL-2011, which were, in turn, 
%% based on the style files for ACL-2010, which were, in turn, 
%% based on the style files for ACL-IJCNLP-2009, which were, in turn,
%% based on the style files for EACL-2009 and IJCNLP-2008...

%% Based on the style files for EACL 2006 by 
%%e.agirre@ehu.es or Sergi.Balari@uab.es
%% and that of ACL 08 by Joakim Nivre and Noah Smith

\documentclass[11pt]{article}
\usepackage{acl2014}
\usepackage{times}
\usepackage{url}
\usepackage{latexsym}

%\setlength\titlebox{5cm}

% You can expand the titlebox if you need extra space
% to show all the authors. Please do not make the titlebox
% smaller than 5cm (the original size); we will check this
% in the camera-ready version and ask you to change it back.


\title{Gender Bias in Fairy Tales}

\author{
    Lisa van Gelderen \\
  {\tt Studentnmr@uva.nl} \\\And
    Alexia Muresan \\
  {\tt Studentnmr@uva.nl} \\\And
    Zoë Prins \\
  {\tt Studentnmr@uva.nl} \\
    Victor van der Sman \\
  {\tt 12943886@uva.nl} \\}

\date{}

\begin{document}
\maketitle
\begin{abstract}
  In this paper will be research into  gender bias in fairy tales. Fairy tales can give children important messages about gender roles, even though these messages might be based on outdated stereotypes [1]. Gender bias in fairy tales might include more focus on beauty in female characters and less on intelligence, and portraying female characters as passive agents [1]. A lot of children grow up hearing fairy tales before bed, or they learn reading by virtue of fairy tales. As such, these stories can have a big impact on the forming of opinions in young children. A lot of the stories are meant to teach children certain traditional values and are often very old. It is therefore worth it to investigate the way these stories handle gender roles. In old stories, usually written by men, gender roles can be quite stereotypical and patronizing towards women. The most classic fairy tale template is a great example of this: “A brave prince needs to rescue a beautiful but helpless princess from her perils.”. If we teach children values with stories that contain stereotypical gender roles, these roles will be perpetuated in the next generation. This is why this repository aims to map the way fairy tales deal with gender.
\end{abstract}


\section{Introduction}

Here we will put into context what the paper aims to achieve and why it is relevant. Also here should be what was omitted from the abstract but important to note. 

\section{Related Work}

Here we will give a short summary of contemporary research in the field regarding our topic.

\section{Data set}

He we will put an explanation and background of the dataset we used for our research

\subsection{Data exploration}

Here we will put some interesting results we got from exploring the data

\section{Method}

Here we will put how we did our research.

\subsection{models}

Here we explain the models we used and how we trained them 

\section{Results}

Here we put the results we get, think of tables, interesting finds and graphs

\section{Conclusion}

Answer to our research question will be here, also here will be some conclusion we can make from the results we got. 

\section{Discussion}

Here we can put some information on how we may have made mistakes and some more personal interpretation of the results. 

\section{Future Work}

Here we can give ideas on how our work can be expended on by researchers in the future that can use the information we found to do the things we did not have the resources or time to look into. 



\begin{thebibliography}{}

\bibitem[\protect\citename{Aho and Ullman}1972]{Aho:72}
Paula Vaz Lobo, and David~M. de Matos
\newblock 2010.
\newblock {\em Fairy Tale Corpus Organization Using Latent Semantic Mapping and an Item-to-item Top-n Recommendation Algorithm}.
\newblock European Language Resources Association (ELRA), Malta.


\end{thebibliography}

\end{document}